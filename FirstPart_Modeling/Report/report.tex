\documentclass[a4paper,11pt]{article}

\usepackage{graphicx} % Required for inserting images
\usepackage[T1]{fontenc}
\usepackage[latin1]{inputenc}
\usepackage{graphicx}
\usepackage{titlesec}
\usepackage{xcolor}
\usepackage{amsfonts}
\usepackage{wrapfig}
\usepackage{amssymb}

\usepackage{glossaries}
\newtheorem{theorem}{Theorem}
\newtheorem{lemma}{Definition}
\newtheorem{proposition}{Proposition}
\newtheorem{proof}{Proof}
\usepackage{tikz}
\tikzstyle{mybox} = [draw=black, thin, rectangle, rounded corners, inner ysep=5pt, inner xsep=5pt, fill=orange!20]

\usepackage[a4paper, top=2cm , bottom=2cm , right=2cm , left=2cm ]{geometry}

\begin{document}
    
    \title{
        \linespread{0.7}
        \vspace{-2.5cm}
        \textbf{
        {\large{MODELING AND CONTROL OF CYBER-PHYSICAL SYSTEMS}}\\
        %{\Large{Project Activity Report}}
        {\Large{Note sulla stesura del Report}}
        \small{(PART I: MODELING)}
        }
    }
    
    \author{
        \textit{ 
        Lorenzo AGHILAR (334086),
        Carlo MIGLIACCIO (332937), 
        Federico PRETINI (329152)}
    }
    \date{}
    \clearpage\maketitle
    \thispagestyle{empty}

<<<<<<< Updated upstream
    \noindent
\textbf{Deadline}
    {\color{red}Entro il 28/06/2024}\\
\noindent
\textbf{Numero di pagine(max)}  4-5 pagine

\section*{Task \#1: IST Algorithm (Carlo)}
\subsection*{Introduzione}
\begin{itemize}
    \itemsep0em
    \item Breve introduzione su ottimizzazione sparsa
    \item $\ell_0$ e sua approssimazione $\ell_1$ 
\end{itemize}
\subsection*{Algoritmo}
\begin{itemize}
    \itemsep0em
    \item Shrinkage/Thresholding operator
    \item ISTA e cenni sulla sua derivazione
\end{itemize}
\subsection*{Risultati}
\begin{itemize}
    \item Grafico Tasso di successo vs numero di sensori $q$; (potrebbe essere un'idea: slider su LiveScript \#sensori)
    \item Tabella con Valore di $\tau$, Tempi di convergenza, valore di $\lambda$, Tasso di successo (evidenziando min, mean, max);
    \item slider per valori di $\tau$ (influisce sul tempo di convergenza) e $\lambda$ (influisce sulla sparsità della soluzione trovata, supporto);
    \item Commento su: 
    \begin{enumerate}
        \item Il risultato che ottengo \`e lo stesso? (NO $\to$ bias, errore)
        \item Il supporto lo riesco sempre a recuperare?
    \end{enumerate}
\end{itemize}


\section*{Task \#2: Localization under sparse sensor attacks (Lorenzo)}
\subsection*{Introduzione}
\begin{itemize}
    \item LASSO, sparse optimization e CPSs
    \item Perch\'e uso l'ottimizzazione sparsa per la SSE di CPSs?
    \item Qualche commento sul setting centralizzato e sull'utilizzo del Fusion Center (ricordare: non ci sono attacchi al fusion center...)
\end{itemize}
\subsection*{Algoritmo}
\begin{itemize}
    \item Hyperparameters utilizzati
    \item Estensione del problema del LASSO (pesi $\lambda$ differenti)
    \item Algoritmo ISTA per la risoluzione del LASSO
\end{itemize}
\subsection*{Risultati}
\begin{itemize}
    \item Confronto AWARE vs UNAWARE su:
    \begin{itemize}
        \item Tasso di rilevamento attacchi
        \item Accuratezza della stima
    \end{itemize}
\end{itemize}

\section*{Task \#3: Localization under sparse sensor attacks (Lorenzo)}
\subsection*{Introduzione}
\begin{itemize}
    \item Indoor localization e RSS fingerprinting
    \item Perch\'e usiamo la sparsit\`a? Cell-grid discretization...
\end{itemize}
\subsection*{Algoritmo}
\begin{itemize}
    \item Hyperparameters utilizzati
    \item Weighted LASSO: $\lambda_1, \lambda_2$ per la soluzione
    \item K-NN: Svantaggi etc...
\end{itemize}
\subsection*{Risultati}
\begin{itemize}
    \item Grafico room con sensori e target nei due casi AWARE e UNAWARE \textsf{(idea: sul Live Script si potrebbe mettere il Menu a tendina per scegliere AWARE/UNAWARE)}
\end{itemize}

\section*{Task \#4: Dynamic SSE (Federico)}
\subsection*{Introduzione}
\begin{itemize}
    \item Breve descrizione del setting dinamico
\end{itemize}
\subsection*{Algoritmo}
\begin{itemize}
    \item Hyperparameters utilizzati
    \item Qualche parola su Online Gradient Descent
    \item Qualche parola su Sparse Observer 
    \item In riferimento agli iperparametri utilizzati mostrare che gli autovalori della matrice $A-L_gC$ siano adeguati (stima asintotica dello stato)
\end{itemize}
\subsection*{Risultati}
\begin{itemize}
    \item Caso base: Snapshot della stanza in momenti particolari (es: stima completamente errata, stima parzialmente corretta, tracking OK...) \textsf{
        (idea: uso del comando \texttt{subplot()} su MATLAB)
    }
    \item (Optional 1) Aware time-variyng attacks: dopo quanto tempo ho convergenza?
    \item (Optional 2) Sensori sotto attacco che cambiano: dopo quanto tempo riesco ad agganciare di nuovo tutto correttamente?
    \item (Optional 3) Qual \`e il limite al numero di sensori? (...qui c'\`e quel problema da risolvere di stima corretta nonostante ci siano tutti e 25 i sensori sotto attacco)
    \item Tabella per confrontare Caso baso e Task opzionali 1 e 2 in termini di: (i) converge/non converge, (ii) Tempo di convergenza (dopo quante iterazioni converge)?
    \item \textsf{idee per Live Script: (i) Scelta aware/unaware, change sensors con checkbox, (ii) Slider con range (min-max) per sensori sotto attacco, numero di target, $T_{max}$...}
\end{itemize}

\section*{Task \#5: Distributed SSE (Carlo)}
\subsection*{Introduzione}
\begin{itemize}
    \item Rimozione fusion center
    \item Vantaggi setting centralizzato e setting distribuito
    \item Consensus
\end{itemize}

\subsection*{Algoritmo}
\begin{itemize}
    \item Distributed ISTA: minimizzazione distribuita del funzionale del LASSO (regularization)
\end{itemize}
\subsection*{Risultati}
\begin{itemize}
    \item Per ogni topologia $Q$
    \begin{enumerate}
        \item Autovalori di Q (rispettano il teorema di Perron/Frobenius)
        \item Consensus si/no
        \item Tempo di convergenza e analisi di esr(Q) per ogni tipologia
        \item Tabella con le informazioni precedenti
        \item Grafico che rappresenti la topologia del grafo (Ricorda: prendi $Q^T$ per usare il comando \texttt{digraph()})
    \end{enumerate}
    \item \textsf{idee per LiveScript: Menu a Tendina per il cambio della topologia...}
\end{itemize}

=======

    \section*{Task \#1: IST Algorithm}
    

    \section*{Task \#2: Secure State Estimation of CPSs}

    \section*{Task \#3: Localization under sparse sensor attacks}
    \newpage
    \section*{Task \#4: Dynamic SSE}
    \subsection*{Introduction}
    In the previous task we performed SSE on a static system, now we leave out the static hypothesis and move on to the dynamic one.
    Recalling the fact that a CPS with some sensors under attack can be described by means of system (1), in the dynamic case the matrix A is no longer the identity matrix but will become more complex.
    \subsection*{Algorithms}
    In order to solve this problem we could think of using least square or Gradient Descent (\textbf{GD}) algorithms but in the former case it might be computationally complex to invert the C-matrix, while in the latter case the algorithm would be too slow to be applied to a dynamic case.
    In order to speed up the GD we can run a single gradient descent step at each k instant, this algorithm has been called Online Gradient Descent or \textbf{OGD} 

    $$\hat{x}(k+1) = A\hat{x}(k) - \tau AC^T[C\hat{x}(k) - y(k)]$$

    Now we can add the attack on the formulation and obtained the so called \textb{augmented observability matrix} $O_t$. From the theory we know that if the matrix A as an eigenvalues equal to 1 the dynamic of the CPS also with constant attack is not observable.
    But thanks to the information about the sparsity of the attack we can develop a SPARSE OBSERVER in order to be able to solve the following problem \\
    \\min....
    \\after a sufficient small number of step T.
    \\The algorithm of the sparse observer is the following:
    
    
    
    
    
    \subsection*{Results}

    \section*{Task \#5: Distributed SSE}
>>>>>>> Stashed changes

\end{document}