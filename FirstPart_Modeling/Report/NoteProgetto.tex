\noindent
\textbf{Deadline}
    {\color{red}Entro il 28/06/2024}\\
\noindent
\textbf{Numero di pagine(max)}  4-5 pagine

\section*{Task \#1: IST Algorithm (Carlo)}
\subsection*{Introduzione}
\begin{itemize}
    \itemsep0em
    \item Breve introduzione su ottimizzazione sparsa
    \item $\ell_0$ e sua approssimazione $\ell_1$ 
\end{itemize}
\subsection*{Algoritmo}
\begin{itemize}
    \itemsep0em
    \item Shrinkage/Thresholding operator
    \item ISTA e cenni sulla sua derivazione
\end{itemize}
\subsection*{Risultati}
\begin{itemize}
    \item Grafico Tasso di successo vs numero di sensori $q$; (potrebbe essere un'idea: slider su LiveScript \#sensori)
    \item Tabella con Valore di $\tau$, Tempi di convergenza, valore di $\lambda$, Tasso di successo (evidenziando min, mean, max);
    \item slider per valori di $\tau$ (influisce sul tempo di convergenza) e $\lambda$ (influisce sulla sparsità della soluzione trovata, supporto);
    \item Commento su: 
    \begin{enumerate}
        \item Il risultato che ottengo \`e lo stesso? (NO $\to$ bias, errore)
        \item Il supporto lo riesco sempre a recuperare?
    \end{enumerate}
\end{itemize}

\section*{Task \#2: Localization under sparse sensor attacks (Lorenzo)}
\subsection*{Introduzione}
\begin{itemize}
    \item LASSO, sparse optimization e CPSs
    \item Perch\'e uso l'ottimizzazione sparsa per la SSE di CPSs?
    \item Qualche commento sul setting centralizzato e sull'utilizzo del Fusion Center (ricordare: non ci sono attacchi al fusion center...)
\end{itemize}
\subsection*{Algoritmo}
\begin{itemize}
    \item Hyperparameters utilizzati
    \item Estensione del problema del LASSO (pesi $\lambda$ differenti)
    \item Algoritmo ISTA per la risoluzione del LASSO
\end{itemize}
\subsection*{Risultati}
\begin{itemize}
    \item Confronto AWARE vs UNAWARE su:
    \begin{itemize}
        \item Tasso di rilevamento attacchi
        \item Accuratezza della stima
    \end{itemize}
\end{itemize}

\section*{Task \#3: Localization under sparse sensor attacks (Lorenzo)}
\subsection*{Introduzione}
\begin{itemize}
    \item Indoor localization e RSS fingerprinting
    \item Perch\'e usiamo la sparsit\`a? Cell-grid discretization...
\end{itemize}
\subsection*{Algoritmo}
\begin{itemize}
    \item Hyperparameters utilizzati
    \item Weighted LASSO: $\lambda_1, \lambda_2$ per la soluzione
    \item K-NN: Svantaggi etc...
\end{itemize}
\subsection*{Risultati}
\begin{itemize}
    \item Grafico room con sensori e target nei due casi AWARE e UNAWARE \textsf{(idea: sul Live Script si potrebbe mettere il Menu a tendina per scegliere AWARE/UNAWARE)}
\end{itemize}

\section*{Task \#4: Dynamic SSE (Federico)}
\subsection*{Introduzione}
\begin{itemize}
    \item Breve descrizione del setting dinamico
\end{itemize}
\subsection*{Algoritmo}
\begin{itemize}
    \item Hyperparameters utilizzati
    \item Qualche parola su Online Gradient Descent
    \item Qualche parola su Sparse Observer 
    \item In riferimento agli iperparametri utilizzati mostrare che gli autovalori della matrice $A-L_gC$ siano adeguati (stima asintotica dello stato)
\end{itemize}
\subsection*{Risultati}
\begin{itemize}
    \item Caso base: Snapshot della stanza in momenti particolari (es: stima completamente errata, stima parzialmente corretta, tracking OK...) \textsf{
        (idea: uso del comando \texttt{subplot()} su MATLAB)
    }
    \item (Optional 1) Aware time-variyng attacks: dopo quanto tempo ho convergenza?
    \item (Optional 2) Sensori sotto attacco che cambiano: dopo quanto tempo riesco ad agganciare di nuovo tutto correttamente?
    \item (Optional 3) Qual \`e il limite al numero di sensori? (...qui c'\`e quel problema da risolvere di stima corretta nonostante ci siano tutti e 25 i sensori sotto attacco)
    \item Tabella per confrontare Caso baso e Task opzionali 1 e 2 in termini di: (i) converge/non converge, (ii) Tempo di convergenza (dopo quante iterazioni converge)?
    \item \textsf{idee per Live Script: (i) Scelta aware/unaware, change sensors con checkbox, (ii) Slider con range (min-max) per sensori sotto attacco, numero di target, $T_{max}$...}
\end{itemize}

\section*{Task \#5: Distributed SSE (Carlo)}
\subsection*{Introduzione}
\begin{itemize}
    \item Rimozione fusion center
    \item Vantaggi setting centralizzato e setting distribuito
    \item Consensus
\end{itemize}

\subsection*{Algoritmo}
\begin{itemize}
    \item Distributed ISTA: minimizzazione distribuita del funzionale del LASSO (regularization)
\end{itemize}
\subsection*{Risultati}
\begin{itemize}
    \item Per ogni topologia $Q$
    \begin{enumerate}
        \item Autovalori di Q (rispettano il teorema di Perron/Frobenius)
        \item Consensus si/no
        \item Tempo di convergenza e analisi di esr(Q) per ogni tipologia
        \item Tabella con le informazioni precedenti
        \item Grafico che rappresenti la topologia del grafo (Ricorda: prendi $Q^T$ per usare il comando \texttt{digraph()})
    \end{enumerate}
    \item \textsf{idee per LiveScript: Menu a Tendina per il cambio della topologia...}
\end{itemize}
