\section*{Task \#4: Dynamic SSE}
    \subsection*{Introduction}
    In the previous task we performed SSE on a static system, now we leave out the static hypothesis and move on to the dynamic one.
    Recalling the fact that a CPS with some sensors under attack can be described by means of system (1), in the dynamic case the matrix A is no longer the identity matrix but will become more complex.
    \subsection*{Algorithms}
    In order to solve this problem we could think of using least square or Gradient Descent (\textbf{GD}) algorithms but in the former case it might be computationally complex to invert the C-matrix, while in the latter case the algorithm would be too slow to be applied to a dynamic case.
    In order to speed up the GD we can run a single gradient descent step at each k instant, this algorithm has been called Online Gradient Descent or \textbf{OGD} 

    $$\hat{x}(k+1) = A\hat{x}(k) - \tau AC^T[C\hat{x}(k) - y(k)]$$

    Now we can add the attack on the formulation and obtained the so called \textbf{augmented observability matrix} $O_t$. From the theory we know that if the matrix A as an eigenvalues equal to 1 the dynamic of the CPS also with constant attack is not observable.
    But thanks to the information about the sparsity of the attack we can develop a SPARSE OBSERVER in order to be able to solve the following problem \\
    \\min....
    \\after a sufficient small number of step T.
    \\The algorithm of the sparse observer is the following:
    
    
    \subsection*{Results}