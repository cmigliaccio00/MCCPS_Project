\documentclass[a4paper, 12pt]{article}

\usepackage{graphicx} % Required for inserting images
\usepackage[T1]{fontenc}
\usepackage[latin1]{inputenc}
\usepackage{graphicx}
\usepackage{titlesec}
\usepackage{xcolor}
\usepackage{amsfonts}
\usepackage{wrapfig}
\usepackage{amssymb}

\usepackage{glossaries}
\newtheorem{theorem}{Theorem}
\newtheorem{lemma}{Definition}
\newtheorem{proposition}{Proposition}
\newtheorem{proof}{Proof}
\usepackage{tikz}
\tikzstyle{mybox} = [draw=black, thin, rectangle, rounded corners, inner ysep=5pt, inner xsep=5pt, fill=orange!20]

\usepackage[a4paper, top=2cm , bottom=2cm , right=2cm , left=2cm ]{geometry}

\begin{document}
    
    \title{
        \vspace{-2cm}
        \textbf{
        {\large{MODELING AND CONTROL OF CYBER-PHYSICAL SYSTEMS}}\\
        {\huge{Project activity report (some ideas...)}}\\
        \normalsize{(PART I: MODELING)}
        }
    }
    
    \author{
        \textit{    
        Lorenzo AGHILAR,
        Carlo MIGLIACCIO, 
        Federico PRETINI }
    }
    
    \clearpage\maketitle
    \thispagestyle{empty}


    %\section*{Task \#1: IST Algorithm}

    %\section*{Task \#2: Secure State Estimation of CPSs}

    %\section*{Task \#3: Localization under sparse sensor attacks}

    %\section*{Task \#4: Dynamic SSE}

    %\section*{Task \#5: Distributed SSE}
\Large{
    \section*{Giusto qualche idea...}
    Dovendo fare sia per la prima che per la seconda parte un report, scrivo qui qualche idea su come strutturarlo.\\
    
    {\huge{
        $\forall \  \textbf{Task}_i, \quad i=1,...,5$
    }}
    
\indent
    \begin{enumerate}
        \item \textbf{Descrizione del problema}
        \begin{itemize}
            \item Qual \`e lo scopo del Task?
        \end{itemize}
        \item \textbf{Breve introduzione teorica}
        \begin{itemize}
            \item Come viene formulato il problema matematicamente?
            \item (eg.) Perch\'e uso il LASSO?
            \item Perch\'e uso un osservatore?
        \end{itemize}
        \item \textbf{Algoritmo}
        \begin{itemize}
            \item implementazione
            \item descrizione degli iperparametri
            \item breve descrizione del codice
        \end{itemize}
        \item \textbf{Risultati ottenuti} $\Longleftarrow$ {\color{red} \textbf{main focus}}
        \begin{itemize}
            \item tuning degli iperparametri
            \item accuratezza, tempo di convergenza
            \item confronti tra pi\`u metodi (varianti)
            \item Perch\`e certi risultati sono stati ottenuti?
            \item \`E quello che mi aspettavo?
        \end{itemize}
    \end{enumerate}

}

\end{document}